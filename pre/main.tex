\documentclass[UTF8]{beamer}

\usepackage[backend=bibtex,sorting=none]{biblatex}
\addbibresource{reference.bib}
\setbeamerfont{footnote}{size=\tiny}

% \input{setup.tex}

\usetheme{Warsaw}
\useoutertheme{miniframes}

\title{AI for System and System for AI}
\author[Conless Pan]{Conless Pan\inst{$\dagger$}}

\institute[SJTU]{
  \inst{$\dagger$}
  ACM Class 2022\\
  Shanghai Jiao Tong University
}

\date{\today}

\begin{document}

\begin{frame}[plain]
  \titlepage
\end{frame}

\begin{frame}
  \tableofcontents
\end{frame}

\section{Introduction}

\begin{frame}{The Rise of Machine Learning}
  \begin{center} 
    \includegraphics[height=150pt]{figure/midjourney_home.png} 
    \includegraphics[height=150pt]{figure/ai_pic.jpg} 
  \end{center}  
\end{frame}

\begin{frame}{The Rise of Machine Learning}
  \begin{center}
    \includegraphics[height=200pt]{figure/chatgpt.png}
  \end{center}
\end{frame}

\begin{frame}{The Rise of Machine Learning}
  \onslide<1->{When we talk about the rise of machine learning, people usually raise these questions:}
  \begin{itemize}
    \item<2-> What is machine learning?
    \item<3-> Do you know its history?
    \item<4-> Why is it so important today?
  \end{itemize}
  \onslide<5->{But I don't want to talk about them, cause I'm not interested about AI.}
\end{frame}

\begin{frame}{Applications of Machine Learning}
  Anyway, AI is a useful tool.
  \begin{itemize}
    \item <2-> Generative AI
    \item <3-> AI for science
    \item <4-> Others
  \end{itemize}
\end{frame}

\section{AI for System}

\begin{frame}{How can AI promote our research of computer system?}
  Let's compare these two games:\footfullcite{smith1998study}
  \begin{center} 
    \onslide<2->{\includegraphics[height=100pt]{figure/2_bit_saturating.jpg}}
    \onslide<3->{\includegraphics[height=100pt]{figure/go_pic.png}}
  \end{center} 
\end{frame}

\begin{frame}{How can AI promote our research of computer system?}
  And it turns out that...\footfullcite{zouzias2021branch}\footfullcite{silver2017mastering}
  \begin{center} 
    \onslide<2->{\includegraphics[height=100pt]{figure/reinforcement_prediction.png}}
    \onslide<3->{\includegraphics[height=100pt]{figure/master_go.png}}
  \end{center} 
\end{frame}

\begin{frame}{How can AI promote our research of computer system?}
  Their similarities:
  \begin{enumerate}
    \item<2-> Decision-making problems
    \item<3-> A finite state given as input
    \item<4-> A simple decision required as output
  \end{enumerate}

  \onslide<5->{The role that AI plays: looking for a fitting function $y=f(x)$, which calculates the correct/best $y$ with given $x$.}
\end{frame}

\begin{frame}{How does AI perform its job?}
  The perceptron was introduced at first:\footfullcite{jimenez2001dynamic}
  \begin{center} 
    \onslide<2->{\includegraphics[height=120pt]{figure/perceptron.jpg}}
    \onslide<3->{\includegraphics[height=120pt]{figure/bp_structure.png}}
  \end{center}
\end{frame}

\begin{frame}{Applications}
  AMD Zen microarchitecture, the start of "AMD YES".
  \begin{center}
    \includegraphics[height=150pt]{figure/amd_zen_bp.jpg}
  \end{center}
\end{frame}

\begin{frame}{Advanced version}
  Advanced neural network is also being introduced to many subfields of computer system\dots
\end{frame}

\begin{frame}{Reinforcement is all you need}
  Let's go over the basic concept of reinforcement learning.
  \begin{itemize}
    \item <2-> A virtual agent who makes decisions
    \item <3-> A state space $S=\{s_i\}$
    \item <4-> An action space $A=\{a_i\}$
    \item <5-> Rewards $r_{a_i|s_i}$
  \end{itemize}
  \onslide<6-> {
    When agent receives a reward or a feedback, it updates the estimation \begin{align*}
      \mathbb{E} (r_{a_i|s_i})
    \end{align*}
    or under some situation the probability \begin{align*}
      \mathbb{P} (r_{a_i|s_i}>0)
    \end{align*}
  }
\end{frame}

\begin{frame}{Applications}
  The problems that reinforcement learning can deal with:
  \begin{itemize}
    \item Heuristic
    \item Empirical
  \end{itemize}
\end{frame}

\begin{frame}{Database Management System}
  The configuration of the knobs\footfullcite{van2017automatic}
  \begin{center}
    \includegraphics[height=75pt]{figure/dbms_knobs.png}
  \end{center}
  \begin{center}
    \includegraphics[height=120pt]{figure/auto_dbms.png}
  \end{center}
\end{frame}

\begin{frame}{Operating System}
  The management of page table index\footfullcite{margaritov2018virtual}
  \begin{center}
    \includegraphics[height=150pt]{figure/page_index.png}
  \end{center}
\end{frame}

\begin{frame}{Applications}
  Anywhere we use heuristic to make a decision can be replaced by ml!\footfullcite{dean2017machine}
  \begin{itemize}
    \item <2-> Compilers: instruction scheduling, register allocation, loop nest parallelization strategies, ...
    \item <2-> Networking: TCP window size decisions, backoff for retransmits, data compression, ...
    \item <2-> Operating systems: process scheduling, buffer cache insertion/replacement, file system prefetching, ...
    \item <2-> Job scheduling systems: which tasks/VMs to co-locate on same machine, which tasks to pre-empt, ...
    \item <2-> ASIC design: physical circuit layout, test case selection, ...
  \end{itemize}
\end{frame}

\begin{frame}{Framework for RL in Sys}
  There are some frameworks for these systems\footfullcite{mao2019park}:
  \begin{center}
    \includegraphics[height=180pt]{figure/park_project.png}
  \end{center}
\end{frame}

\section{System for AI}

\begin{frame}{How can our works on system promote research of AI?}
  Large models are conquering ml\dots
  \begin{center}
    \includegraphics[height=90pt]{figure/nlp_large_models.png}
    \includegraphics[height=90pt]{figure/cv_large_models.png}
  \end{center}
\end{frame}

\begin{frame}{Problems in system}
  With the growth of size, problems occurred.
  \begin{itemize}
    \item <2-> Low speed
    \item <3-> Limited memory
    \item <4-> Popularization
  \end{itemize}
\end{frame}

\begin{frame}{Improvements}
  Speed can be achieved with loss of precision and generality
  
  \begin{columns}
    \column{0.5\textwidth}
    \centering Reduced precision
    \column{0.5\textwidth}
    \centering \includegraphics[height=4\baselineskip]{figure/reduced_prediction.png}
  \end{columns}

  \begin{columns}
    \column{0.5\textwidth}
    \centering Specific operations
    \column{0.5\textwidth}
    \centering \includegraphics[height=4\baselineskip]{figure/conv_oper.png}
  \end{columns}
\end{frame}

\begin{frame}{Improvements}
  Thus we have Nvidia GPUs with CUDA units and tensor cores, Google TPU and Apple NPU.
  \begin{center}
    \includegraphics[height=4\baselineskip]{figure/nvidia_gpu.png}
    \hfill
    \includegraphics[height=4\baselineskip]{figure/google_tpu.png}
    \hfill
    \includegraphics[height=4\baselineskip]{figure/apple_m1_npu.png}
  \end{center}
\end{frame}

\begin{frame}{Improvements}
  In traditional data storage and computing, we distribute data/tasks to multiple machines for larger storage size and better speed.
  \includegraphics[height=150pt]{figure/distributed_system.png}
\end{frame}

\begin{frame}{Improvements}
  The same idea can also be applied into ml
  \begin{itemize}
    \item <2-> Distributed machine learning system (parallel compute / parallel data)
    \item <3-> Computing device placement
    \item <4-> Other topics in traditional distributed sys (communication, consistency)
  \end{itemize}
\end{frame}

\begin{frame}{Convenience}
  Programmers don't want to build their projects from every detail. 
  
  \begin{center}
    \includegraphics[height=45pt]{figure/cpp_stl.png}
    \hfill
    \includegraphics[height=45pt]{figure/qt_logo.png}
    \hfill
    \includegraphics[height=45pt]{figure/directx.jpeg}
  \end{center}
  
  When writing cpp programs, it's convenient for us to use those frameworks and libraries.
  
  AI programmers also need them.
\end{frame}

\begin{frame}{Frameworks}
  \begin{minipage}[b]{0.6\textwidth}
    \includegraphics[height=150pt]{figure/dl_framework.png}
  \end{minipage}
  \begin{minipage}[b]{0.3\textwidth}
    \includegraphics[height=80pt]{figure/nvidia_cuda.png}
    \includegraphics[height=20pt]{figure/tvm_logo_small.png}
  \end{minipage}
\end{frame}

\begin{frame}{Frameworks}
  \includegraphics[height=200pt]{figure/cnn_pytorch.png}
\end{frame}

\section{Prospects}

\begin{frame}{Higher level ideas}
  \includegraphics[height=180pt]{figure/co_design.png}
\end{frame}

\begin{frame}{Research}
  What is a good sys4ml/ml4sys research?
  \begin{itemize}
    \item<2-> Should be both good AI and systems research
      \begin{itemize}
        \item<3-> Provides insights to both communities
      \end{itemize}
    \item<4-> Leverages understanding of both domains
    \item<5-> I don't like adjust those parameters
  \end{itemize}
\end{frame}

\begin{frame}
  \Huge{\centerline{Thank you!}}
\end{frame}

\end{document}